\documentclass{article}

\usepackage{titlesec}
\usepackage{titling}
\usepackage{scrextend}
\usepackage[margin=0.5in]{geometry}
\usepackage{multicol}

\pagenumbering{gobble}

\titleformat{\section}
{\huge\bfseries}
{}
{0em}
{}[\titlerule]

\titleformat{\subsection}[runin]
{\bfseries}
{}
{0em}
{}

\titleformat{\subsubsection}[runin]
{\bfseries}
{}
{0em}
{}[:]

\titlespacing{\section}
{0em}{0.05em}{0.05em}
\titlespacing{\subsection}
{0em}{0em}{0em}
\titlespacing{\subsubsection}
{0em}{0.1em}{0.25em}

\renewcommand{\maketitle}{

    
    {\noindent\huge\bfseries\theauthor} \hfill \textit{Last Updated: \today}

    \vspace{0.25em}

    \noindent\textbf{portfolio:} https://rohitmusti.github.io/ | \textbf{email:} rohit.musti.rm@gmail.com | \textbf{github:} https://github.com/rohitmusti
    
}

\begin{document}

\title{Resume}
\author{Rohit Musti}

\maketitle
\vspace{-.35em}

\section{Work Experience}
\vspace{.15em}

\subsection{American Forests}, \textit{Director of Software and Data Engineering} \hfill \textit{40 hours/week, January '22 - Present}

\noindent 
- \textit{Tree Equity Score}: Delivered first-of-its-kind national Tree Equity Score; covered in NYTimes and many local news outlets.
\begin{addmargin}[2em]{0em}% 1em left, 2em right
- Coordinating, planning, and contributing to the maintenance of our existing data infrastructure\\
- Re-architecting and building our Deep Dive Analysis and Plannign tool to better horizontally scale. \\
- Continuing to build our data stories and communication infrastrcture and work through data shorts and integrating data into other communications work.
\end{addmargin}
- \textit{Technology Management:} Architecting and managing dev-ops infrastructure across the organzation. \\
- \textit{In Progress:} analysis and visualization of tree species migration under various climate action scenarios; developing an AI model to derive tree canopy data from satellite images, creating free and open tree canopy data nationwide for the first time.

\subsection{American Forests}, \textit{Senior Manager of Data Science (Software \& Data Engineer)} \textit{40 hours/week, May '20 - January '22}
\noindent 
- \textit{Tree Equity Score}: Delivered first-of-its-kind national Tree Equity Score; covered in NYTimes and many local news outlets.
\begin{addmargin}[2em]{0em}% 1em left, 2em right
- Solely developed data processing pipelines that calculate the Tree Equity Score for over 150,000 neighborhoods in 486 US cities; combining over 25 datasets (2,558 satellite tiles, 14,586 census files, \& several tree canopy datasets).\\
- Developed an user-friendly interactive web map and impact calculator for exploring the Tree Equity Score and a deep-dive analysis tool for urban foresters \& city planners to set equity-focused tree canopy goals and plan the exact parcels that they want to plant trees on (Tree Equity Score Explorer \& Analyzer); over 40,000 users combined.\\
- Exceeded goal of calculating Tree Equity Score for 5 cities, instead delivering it to 486 municipalities.\\
- Wrote and built data stories about the intersections between tree canopy and race, poverty, and health. \\
- Communicated data engineering methods to technical \& non-technical audiences (ranging from peers at a data conference to Detroit city planners).
\end{addmargin}
- \textit{Interactive Web Mapping}: Created web map and database of over 1,400 tree planting \& urban forestry projects going back to 1990. Developed career pathways interactive web map to help users access career opportunities in urban forestry.\\
- \textit{Technology Management:} Helped clean and manage CRM and assisted in the management of its GIS \& IT infrastructure. Collaborated with database stakeholders to improve data quality, enabling more informed fundraising.


\subsection{Hunter College, CUNY}, \textit{Lecturer} \hfill \textit{20 hours/week, November '21 - Present}

\noindent 
- Teaching a 3 credit undergraduate course (in Spring 2022) as a member of the Tech-In-Residence Corps.\\
- Lecturing for 2 hours and 40 minutes a week for 16 weeks \& developing all course material, assignments, and exams


\subsection{Graduate Cryptography} \textit{ Graduate Teaching Assistant} \hfill \textit{5 hours/week, January - May '20}

\noindent - Hold 1.5 weekly office hours for a class of 30 students, as sole TA, and proof read all 5 homework assignments \\
- Grade 5 homework assignments \& take home exam for all 30 students, and write test cases for 2 programming homeworks

\subsection{Graduate Research Work} \textit{ Graduate Student Researcher } \hfill \textit{5 hours/week, August '19 - May '20}

\noindent - Developed a neural net to detect code switches (changes in language/dialect), enabling multi-lingugal NLP models\\
- Implemented a typechecking system language that allowed for further optimizations to be made based on the type analysis

\subsection{Digital Governance Lab} \textit{ Student Instructor} \hfill \textit{5 hours/week, August '19 - May '20}

\noindent - Developed public interest technology curriculum to critically examine the impacts of digital technology (13 seminars/semester) \\
- In the Fall, taught to 8 students (5 tech \& 3 policy), culminating in a critique of senior engineering capstone projects \\
- In the Spring, taught to 6 students (4 tech \& 2 policy), each developing their own public interest tech policy or project


\subsection{Red Hat: AI Center of Excellence} \textit{ Graduate AI Research Intern} \hfill \textit{40 hours/week, May - August '19}

\noindent - Designed internal data processing pipeline to clean and featurize client interaction data for AI Natural Language Processing \\
- Researched NLP question and answering and text generation techniques, working towards a client-facing chat bot\\
- Identified structural issues with the state of the art NLP techniques that block approaches scalable solutions

\subsection{Algorithms} \textit{ Head Teaching Assistant} \hfill \textit{20 hours/week, January - May 2018}

\noindent - Managed team of 5 TAs and organized review sessions twice a week to recap content \\
- Edited 10 homework assignments, designed 2 exams, at least 3 test cases per homework, \& managed auto-grading tools

\subsection{Red Hat: Open Innovation Labs} \textit{ Site Reliability Engineering Intern} \hfill  \textit{40 hours/week, May - August '18}

\noindent - Worked directly with the software reliability engineering team to solve pressing back log items to stabilize dev-ops pipeline \\
- Automated infrastructure deployment of Open Shift and all relevant tooling, saving an estimated 100 hours per client \\
- Built search feature for the Open Practice Library from scratch to increase access to the library \\
- Participated in the Open Innovation Lab’s DevOps Enablement training: learned Agile and DevOps best practices

\subsection{Introduction to Computer Science} \textit{ Teaching Assistant} \hfill \textit{10 hours/week, January - December '17}

\noindent - Led 2 lab group review sessions per semester, graded 20 homework assignments/exams for over 20 students each semester \\
- In first semester, helped over 180 students, most of any other TA

\section{Education}

\setlength{\columnsep}{-17em}
\vspace{-1em}
\begin{multicols}{2}
    \noindent University of Virginia \\ 
    Bachelors of Arts, Computer Science '19 \\
    Masters of Computer Science \textit{(3 + 1)} '20 \\

    \columnbreak 
    \noindent Honors: Jefferson Scholar (full ride merit scholarship), Echols Scholar, Dean's List \\ 
    Focus: Deep Learning, Algorithms, Cryptography, Social Impacts of Technology
\end{multicols}
\vspace{-2.5em}

\section{Skills}
\vspace{.15em}

\subsubsection{Programming Languages}
Python, Java, C++, C, Javascript, R, Rust, Typescript
\subsubsection{Tools}
{\LaTeX}, Pytorch, Tensorflow, Ansible, Docker, NextJS, GDAL, Arcpy, GeoPandas, Mapbox, Git

\end{document}